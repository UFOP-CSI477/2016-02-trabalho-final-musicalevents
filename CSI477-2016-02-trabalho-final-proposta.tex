\documentclass[10pt,a4paper,article]{abntex2}
\usepackage[utf8]{inputenc}
\usepackage{amsmath}
\usepackage{amsfonts}
\usepackage{amssymb}
\usepackage{url}

% Nome dos autores do trabalho
\author{Grupo: Jonathan Paulo de Morais}
\title{CSI477-2016-02 -- Proposta de Trabalho Final}
\begin{document}

	\maketitle

	% Descrever um resumo sobre o trabalho.
	\begin{abstract}
		O objetivo deste documento é apresentar uma proposta para o trabalho a ser desenvolvido na disciplina CSI477 -- Sistemas WEB I. É uma breve descrição sobre o tema que será abordado, bem como o escopo, as restrições e demais questões pertinentes ao contexto.
	\end{abstract}		
	
	% Apresentar o tema.
	\section{Tema}
	
		O trabalho final tem como tema o desenvolvimento de um sistema para contratação/aluguel de músicos/instrumentos para eventos em geral.
	
	% Descrever e limitar o escopo da aplicação.
	\section{Escopo}
	
		Este projeto terá as seguintes funcionalidades: 
		
		(a) cadastro de músicos, profissionais ou não, os quais oferecerão seu serviço para eventos, tais como aniversários, casamentos, shows, freelancers, couvert artísticos em barzinhos ou similares, gravações, entre outros;

		(b) cadastro de outros profissionais do ramo musical os quais podem oferecer serviços de sonoplastia, roadie (carrega instrumentos/equipamentos, preparação e montagem da aparelhagem no palco), etc;

		(c) cadastro de instrumentos musicais para aluguel, sendo que o valor do aluguel varia de acordo com a qualidade do instrumento e sua popularidade entre os usuários;

		(d) cadastro do cliente interessado nas funcionalidades do sistema;
		
		(e) pontuação para mensurar a satisfação do cliente pela contratação dos serviços e produtos;

		(f) similarmente, os clientes serão avaliados quanto integridade dos instrumentos no ato da devolução.

	% Apresentar restrições de funcionalidades e de escopo.
	\section{Restrições}

		Neste trabalho não serão considerados: 
		(a) suporte a venda de instrumentos musicais, uma vez que já existem diversas aplicações para tal fim - o foco será no aluguel;

		(b) não haverá suporte à transações financeiras pelo sistema - deverá ser tratado diretamente com o ofertante do serviço/produto através do contato disponibilizado no sistema

	% Incluir o link do repositório
	\section{Repositório}

		O trabalho final terá como repositório principal o seguinte endereço: \url{https://github.com/UFOP-CSI477/2016-02-trabalho-final-musicalevents}.

	% Referências podem ser incluídas, caso necessário.
	%\section{Referências}

\end{document}